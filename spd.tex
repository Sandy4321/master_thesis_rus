\section{Геометрия Риманова многообразия}

В этом разделе описаны принципы и инструменты геометрии многообразия симметричных положительно определенных матриц (Риманова многообразия.)

\subsection{Риманово многообразие}

\begin{definition}
Римановым называется многообразие $\manM$ с заданным на нем  $m \in \manM$ скалярным произведением $\left< \cdot \right>_\bc$ касательных векторов в касательном пространстве $T_p \manM$, таким, что оно гладко зависит от $p$.
\end{definition}

\begin{definition}
Риманова метрика $g$ на многообразии $\manM$ – семейство всех скалярных произведений на всех касательных пространствах
\end{definition}

\indent Далее будем рассматривать многообразие симметричных положительно определенных (СПО) матриц как пример риманова многообразия. Имеем следующие свойства: 
\begin{lemma}
Свойства пространства $S(n)$:
\begin{itemize}
	\item $\forall \ \bs \in S(n),\ det(\bs) > 0$
	\item $\forall \ \bs \in S(n),\ \bs^{-1} \in S(n)$
	\item $\forall \ (\bs_1, \bs_2) \in S(n)^2,\ \bs_1 \bs_2 \in C(n)$
\end{itemize}
\end{lemma}

\indent СПО матрицы из семейства $S(n)$ всегда диагонализуемы и имеют строго положительные действительные собственные значения. Для СПО матриц в $S(n)$ матричная экспонента $\bs$ задается через собственные значения сингулярно-векторного разложения матрицы $\bs$:
$$ \bs = {\bf U} \ diag(\sigma_1, \ldots, \sigma_n) \ {\bf U}^T, $$
где $\sigma_1 > \sigma_2 > \ldots > \sigma_n$ - собственные значения и ${\bf U}$ - матрица собственных векторов $\bs$. 
\begin{definition}
Матричная экспонента $\bs \in S(n)$: 
 $$ exp(\bs) = {\bf U} \ diag(exp(\sigma_1), \ldots, exp(\sigma_n)) \ {\bf U}^T $$
\end{definition}

\begin{definition}
Логарифм матрицы $\bs \in S(n)$:
$$ log(\bs) = {\bf U} \ diag(log(\sigma_1), \ldots, log(\sigma_n)) \ {\bf U}^T  $$ \\
\end{definition}
\indent Взятие логарифма матрицы $\bs \in S(n)$ является обратной операцией к взятию матричной экспоненты.

\begin{lemma}
Свойства пространства $S(n)$, связанные с матричными экспонентой и логарифмом:
\begin{itemize}
	\item $\forall \ \bs \in S(n),\ log(\bs) \in Sym(n)$
	\item $\forall \ \bc \in Sym(n),\ exp(\bc) \in S(n)$
	    
\end{itemize}
\end{lemma}

\subsection{Риманова метрика}
Пространство СПО матриц $S(n)$ – это риманово многообразие $\setM$, следовательно, оно дифференцируемо. Производная матрицы $\bs$ на многообразии лежит в векторном пространстве $T_{\bs}$, являющимся касательным пространством к этой точке. Касательное пространство лежит в пространстве $Sym(n)$. Многообразие и касательное пространство имеют размерность $m = n(n+1)/2$ \cite{faraut1994analysis} \\
\indent Для каждого касательного пространства определено скалярное произведение $\left< \cdot \right>_\bc$, гладко меняющееся от точки к точке. Натуральная метрика на многообразии СПО матриц определена локальным скалярным произведением: 
\begin{equation} \label{nmetr}
	 \inprod{C_1}{C_2}{\bs} = Tr(C_1\bs^{-1}C_2\bs^{-1})
\end{equation}

\indent Через скалярное произведение можем задать норму касательных векторов: \begin{equation} \label{nnorm}
	 \norm{\bc}{\bs}^2 = \inprod{\bc}{\bc}{\bs} = Tr(\bc\bs^{-1}\bc\bs^{-1})
\end{equation}

\subsection{Геодезические в римановом пространстве}
Обозначим $ \Gamma(t): \ [a,b] \rightarrow S(n) : \bs^i=\bs^i(t), i=\set{1, \ldots, n}, a \leq t \leq b$ - произвольная кривая риманова многообразия $(\manM, g)$. В любой точке этого пути касательный вектор определяется как: 
$$ v(t) =(\dot{\bs}^1(t), \ldots, \dot{\bs}^n(t) )$$
$$ |v(t)|_\bs = \sqrt{\left< v,v \right>_g\rvert_\bc} $$
Длина этой кривой определяется как: 
\begin{equation}
	L(\Gamma(t)) = \int_{a}^{b} \norm{\dot{\Gamma}(t)}{\Gamma(t)} dt = \int_{a}^{b} \sqrt{g_{ik}(\bs)\dot{\bs}^i\dot{\bs}^k} dt, \\
	g_{ik} = \left< \frac{\partial}{\partial \bs^i}, \frac{\partial}{\partial \bs^k} \right >
\end{equation}
где используется норма, определенная в предыдущем разделе формулой \eqref{nnorm}. Путь минимальной длины, соединяющий две точки многообразия, называется геодезической, и риманово расстояние между двумя точками определяется как длина этого пути. Натуральная метрика \eqref{nmetr} задает геодезическое расстояние.

\begin{equation} \label{geodd}
	\delta_{spd}(\bs_1,\bs_2) = \norm{log(\bs_1^{-1}\bs_2)}{F} = \Big{[} \sum_{i=1}^{n} log^2 \lambda_i \Big{]}^{1/2},
\end{equation} 

где $\lambda_1, \ldots, \lambda_n$ - действительные собственные значения $\bs_1^{-1}\bs_2$. Главные свойства риманова геодезического расстояния:
\begin{itemize}
	\item $\delta_{spd}(\bs_1,\bs_2) = \delta_{spd}(\bs_2, \bs_1)$
	\item $\delta_{spd}(\bs_1^{-1},\bs_2^{-1}) = \delta_{spd}(\bs_1,\bs_2)$
	\item $\delta_{spd}(\bw^T\bs_1\bw,\bw^T\bs_2\bw) = \delta_{spd}(\bs_1,\bs_2) \ \forall \ \bw \in Gl(n)$
\end{itemize}


\subsection{Экспоненциальная проекция}
Для каждой точки $\bs \in S(n)$ можно задать касательное пространство $\tau_C\mathcal{M}$, образованное множеством векторов, касательных к $\bs$. Каждый касательный вектор $\bc_i$ может рассматриваться как производная в точке $t=0$ от геодезической $\Gamma_i(t)$ между $\bs$ и экспоненциальной проекцией $\bs_i=Exp_\bs(\bc_i)$ \\
% \begin{figure}[h]
% 	\centering
% 	\includegraphics[width=0.7\textwidth]{./img/projection}
% 	\caption{Многообразие $\mathcal{M}$ и соответствующее касательное пространство $\tau_C\mathcal{M}$ в точке $\bc$}
% \end{figure} \\
\indent Экспоненциальная проекция определяется как:
\begin{equation}
	Exp_\bs(\bc_i) = \bs_i = \bs^{1/2}exp(\bs^{-1/2}\bc_i\bs^{-1/2})\bs^{1/2}
\end{equation}
\indent Обратная проекция определяется как логарифмическая проекция вида:
\begin{equation} \label{logpr}
	Log_\bs(\bs_i) = \bc_i = \bs^{1/2}log(\bs^{-1/2}\bs_i\bs^{-1/2})\bs^{1/2}
\end{equation}
\indent В терминах проекции можно ввести эквивалетное определение риманового расстояния:
$$ \delta_R(\bs_1,\bs_2) = \norm{Log_\bs(\bs_i)}{\bs} = \norm{\bc_i}{\bs} = \\ $$
$$	= \norm{upper(\bs^{-1/2}Log_\bs(\bs_i)\bs^{-1/2}}{2} = \norm{c_i}{2}, $$
где $upper(\cdot)$ - оператор, оставляющий верхнетреугольную часть симметричной матрицы и векторизующий ее, добавляя единичный вес диагональным элементам и $\sqrt{2}$ вес не-диагональным. $c_i$ здесь - $m$-мерный вектор $upper(\bs^{-1/2}Log_\bs(\bs_i)\bs^{-1/2}$ нормализованного касательного пространства.