\chapter{Заключение}
\indent В этой работе были рассмотрены различные подходы к классификации нейрологических данных на примере данных электроэнцефалографии, функциональной и структурной магнитно-резонансной томографии. Также был рассмотрен математический аппарат, используемый в алгоритмах для этой задачи, а именно концепции геометрии Римана, геометрия многообразия симметричных положительно определенных (СПО) матриц и геометрия многообразия симметричных положительно полуопределенных (СППО) матриц. Геометрия многообразия СПО матриц хорошо изучена и является сильным инструментом, использовавшимся в большом количестве работ по этой теме. Однако геометрия пространства СППО матриц изучена гораздо меньше и до сих пор не применялясь в практических задачах машинного обучения, в частности в задаче классификации данных МРТ.\\
\indent В первой главе работы был дан обзор задачи, кратко описаны основные подходы и объяснена важность задачи. Во второй главе были разобраны техники неизвазивного исследования головного мозга, а именно функциональная и структурная МРТ, а также введено понятие графа функциональных либо структурных связей мозга – коннектома. В третьей главе был выполнен обзор существующих работ по задаче классификации снимков МРТ и кратко представлены полученные в них результаты. \\
\indent В четвертой главе подробно описан используемый в этой работе математический аппарат: в первой части были введены общие определения и обозначения; во второй части разобрана геометрия Римана на примере многообразия СПО матриц: описано многообразие, существующая на нем метрика, геодезические на многообразии и процесс проецирования на касательное пространство; в третьей части введена теория, описывающая многообразие СППО матриц: описана метрика на этом многообразии, горизонтальное пространство к этому многообразию, разобран процесс построения геодезических кривых между СППО матрицами и задана метрика сходства матриц, условно названная расстоянием. \\
\indent В пятой главе подробно рассмотрены два основных подхода в классификации СПО матриц, а именно, ядерные методы на основе расстояния между СПО матрицами и класификация в касательном пространстве, важным шагом которой является проекция на касательное пространство исходных СПО матриц. В шестой главе предложено два метода решения задачи классификации коннектомов, оба из которых первым шагом имеют конвертацию коннектомов в СППО матрицы с помощью преобразования Лапласа: первый метод основан на снижении размерности данных с помощью алгоритма Isomap и классификации данных с помощью логистической регрессии, второй является принципиально новым подходом, не использовавшимся ранее, и основан на примерении геометрии пространства СППО матриц, описанной в пятой главе, для построения положительно определенного ядра с помощью расстояния между СППО матрицами на изначальном многообразии. \\
\indent В главе семь описаны данные, на которых производились вычисления, и постановка экспериментов. Было рассмотрено два эксперимента: влияние ранга матрицы на точность классификации (для этого было произведено искусственное снижение ранга СППО матриц) и сравнение результатов классификации двух существовавших и двух предложенных методов. В главе восемь произведен анализ результатов, показавших, что предложенные алгоритмы успешно решают задачу классификации снимков здоровых людей и людей с болезнью Альцгеймера. Также важным результатом является то, что предложенный алгоритм на основе расстояния на многообразии СППО матриц показал результаты, значительно превосходящие результаты остальных алгоритмов, в задаче классификации снимков здоровых людей и снимков людей с ранними когнитивными нарушениями, что позволяет решать задачу ранней диагностики нейродегенеративнх заболеваний.